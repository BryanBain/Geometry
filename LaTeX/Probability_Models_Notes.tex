\documentclass{article}
\usepackage{amsmath, sfmath, multicol, tkz-euclide, array, enumerate, tcolorbox, tabularray, tipa}
\renewcommand{\familydefault}{\sfdefault}
\setlength{\parindent}{0cm}
\pagestyle{empty}
\usepackage[left=1in, top=0.5in, right=1in, bottom=0.5in]{geometry}
\tikzset{>=stealth, label style/.append style={font=\footnotesize}}
\tcbset{colback=white}

\newcounter{example}[section]
\newenvironment{example}[1][]{\refstepcounter{example}\par\medskip
   {\color{red}\textbf{Example~\theexample. #1}}}{\medskip}

\newcommand{\arc}[1]{%
    \setbox9=\hbox{#1}%
    \ooalign{\resizebox{\wd9}{\height}{\texttoptiebar{\phantom{A}}}\cr#1}}

\begin{document}

\section*{Probability Models}

\begin{tcolorbox}[colframe=orange!70!white, coltitle=black, title=\textbf{Today I Can}]
\begin{enumerate}
    \item Construct and use probability models.
\end{enumerate}
\end{tcolorbox}
\smallskip

\begin{tcolorbox}[colframe=black!20!white, opacitybacktitle=0.1, coltitle=black, title=\textbf{Two-Way Frequency Table (a.k.a Contingency Table)}]
A display of the frequencies of data in 2 or more different categories.
\end{tcolorbox}
\smallskip

\begin{example}
The table shows data about students grade level and whether or not they work part-time. Find each probability.

\begin{center}  \setlength{\extrarowheight}{4pt}
    \begin{tabular}{c|c|c|c}
            & \textbf{Part time job} & \textbf{No part time job} & \textbf{Totals} \\[4pt] \hline 
        \textbf{Junior} & 19 & 31 & \\[4pt] \hline 
        \textbf{Senior} & 44 & & 50 \\[4pt] \hline 
        \textbf{Totals} &  & 37 & 100
    \end{tabular}
\end{center}
\smallskip 

\begin{enumerate}[(a)]
\begin{multicols}{2}
    \item Junior and does not work
    \item Senior and does work
\end{multicols}
\end{enumerate}
\end{example}

\vspace{1in}

\begin{tcolorbox}[colframe=black!20!white, opacitybacktitle=0.1, coltitle=black, title=\textbf{Conditional Probability}]
The probability that an event will occur given that another event has already occurred. \newline 

\begin{itemize}
    \item The probability event $B$ occurs given that event $A$ has occurred: $P(B|A)$
    \item The probability event $A$ occurs given that event $B$ has occurred: $P(A|B)$
    \item Typically, whatever follows the key words and phrases \textit{if}, \textit{given that}, or \textit{suppose} is the value that goes in the denominator.
\end{itemize}
\end{tcolorbox}

\newpage 

\begin{example}
Respondents of a poll were asked whether they were for, against, or had no opinion about a bill before the state legislature that would increase minimum wage.

\begin{center}
    \begin{tabular}{c|c|c|c|c}
        \textbf{Age Group} & \textbf{For} & \textbf{Against} & \textbf{No Opinion} & \textbf{Total} \\ \hline 
        \textbf{18-29} & 310 & 50 & 20 & 380 \\ \hline 
        \textbf{30-45} & 200 & 30 & 10 & 240 \\ \hline 
        \textbf{46-60} & 120 & 20 & 30 & 170 \\ \hline 
        \textbf{Over 60} & 150 & 20 & 40 & 210 \\ \hline 
        \textbf{Total} & 780 & 120 & 100 & 1000
    \end{tabular}
\end{center}

\begin{enumerate}[(a)]
    \item What is the probability that a randomly selected person is over 60 years old given that the person had no opinion?
    \vfill 
    \item What is the probability that a randomly selected person is 30-45 years old if the person is in favor?
    \vfill
    \item Suppose someone who is against the bill is selected, what is the probability that they are 18-29?
    \vfill
    \item What is the probability that a randomly selected person is \emph{not} 18-29, given that the person is in favor of the bill?
\end{enumerate}
\end{example}
\vfill

\begin{example}
A company has 150 sales representatives. Two months after a sales seminar, the company vice-president made the table based on sales results.

\begin{center}
    \begin{tabular}{c|c|c|c}
            & \textbf{Attended} & \textbf{Did not attend} & \textbf{Total} \\ \hline 
        \textbf{Increased Sales} & 0.48 & 0.02 & 0.5 \\ \hline 
        \textbf{No increased sales} & 0.32 & 0.18 & 0.5 \\ \hline 
        \textbf{Total} & 0.8 & 0.2 & 1
    \end{tabular}
\end{center}

\begin{enumerate}[(a)]
    \item What is the probability that someone who attended the seminar had an increase in sales?
    \vfill
    \item What is the probability that a randomly selected sales representative, who did not attend the seminar, did not see an increase in sales?
    \vfill
\end{enumerate}
\end{example}

\end{document}
