\documentclass{article}
\usepackage{amsmath, sfmath, multicol, tkz-euclide, array, enumerate, tcolorbox, tabularray}
\renewcommand{\familydefault}{\sfdefault}
\setlength{\parindent}{0cm}
\pagestyle{empty}
\usepackage[left=1in, top=0.5in, right=1in, bottom=0.5in]{geometry}
\tikzset{>=stealth}
\tcbset{colback=white}

\newcounter{example}[section]
\newenvironment{example}[1][]{\refstepcounter{example}\par\medskip
   {\color{red}\textbf{Example~\theexample. #1}}}{\medskip}

\begin{document}

\section*{Ratios and Proportions}

\begin{tcolorbox}[colframe=orange!70!white, coltitle=black, title=\textbf{Today I Can}]
\begin{enumerate}
    \item Write ratios and solve proportions.
\end{enumerate}
\end{tcolorbox}
\smallskip

\begin{tcolorbox}[colframe=black!20!white, opacitybacktitle=0.1, coltitle=black, title=\textbf{Ratio}]
A comparison of two quantities by division. Denoted by
\[
\frac{a}{b}, \quad a:b, \quad   \text{and}  \quad   a\text{ to }b
\]
\end{tcolorbox}
\bigskip 

\begin{example}
\begin{enumerate}[(a)]
    \item Members of the school band are buying pots of tulips and pots of daffodils to sell at their fundraiser. They plan to buy 120 pots of flowers. The ratio
\[
\frac{\text{number of tulip pots}}{\text{number of daffodil pots}}
\]
will be $\frac{2}{3}$. How many pots of each type of flower should they buy?

\vspace{1in}

    \item The measures of two supplementary angles are in the ratio 1 : 4. What are the measures of the angles?

\vspace{1in}
\end{enumerate}
\end{example}

\begin{tcolorbox}[colframe=black!20!white, opacitybacktitle=0.1, coltitle=black, title=\textbf{Extended Ratio}]
A comparison of three (or more) numbers. Denoted by $a$ : $b$ : $c$
\end{tcolorbox}

\begin{example}
\begin{enumerate}[(a)]
    \item The lengths of the sides of a triangle are in the extended ratio 3 : 5 : 6. The perimeter of the triangle is 98 in. What is the length of the longest side?
    \vspace{1in}
    \item The angles of a triangle are in the ratio 2 : 3 : 4. What are the measures of the angles?
\end{enumerate}
\end{example}

\newpage

\begin{tcolorbox}[colframe=black!20!white, opacitybacktitle=0.1, coltitle=black, title=\textbf{Proportion}]
An equation with equal ratios.
\end{tcolorbox}

\begin{example}
Solve each.
\begin{multicols}{3}
\begin{enumerate}[(a)]
    \item $\frac{6}{x}=\frac{5}{4}$  
    \item $\frac{9}{2}=\frac{a}{14}$
    \item $\frac{y+4}{9}=\frac{y}{3}$  
\end{enumerate}
\end{multicols}
\end{example}


\end{document}
