\documentclass{article}
\usepackage{amsmath, sfmath, multicol, tkz-euclide, array, enumerate, tcolorbox, tabularray, tipa}
\renewcommand{\familydefault}{\sfdefault}
\setlength{\parindent}{0cm}
\pagestyle{empty}
\usepackage[left=1in, top=0.5in, right=1in, bottom=0.5in]{geometry}
\tikzset{>=stealth, label style/.append style={font=\footnotesize}}
\tcbset{colback=white}

\newcounter{example}[section]
\newenvironment{example}[1][]{\refstepcounter{example}\par\medskip
   {\color{red}\textbf{Example~\theexample. #1}}}{\medskip}

\newcommand{\arc}[1]{%
    \setbox9=\hbox{#1}%
    \ooalign{\resizebox{\wd9}{\height}{\texttoptiebar{\phantom{A}}}\cr#1}}

\begin{document}

\section*{Conditional Probability Formulas}

\begin{tcolorbox}[colframe=orange!70!white, coltitle=black, title=\textbf{Today I Can}]
\begin{enumerate}
    \item Understand and calculate conditional probabilities.
\end{enumerate}
\end{tcolorbox}
\smallskip

\begin{tcolorbox}[colframe=black!20!white, opacitybacktitle=0.1, coltitle=black, title=\textbf{Conditional Probability Formula}]
For any two events $A$ and $B$, the probability of $B$ occurring, given that event $A$ has occurred is
\[
P(B|A) = \frac{P(A \text{ and } B)}{P(A)} \quad \text{where } P(A) \neq 0
\]
\end{tcolorbox}
\smallskip

\begin{example}
In a study designed to test the effectiveness of a new drug, half of the volunteers received the drug. The other half of the volunteers received a placebo (doesn't contain medication). The probability of a volunteer receiving the drug and getting well was 45\%. 

\begin{enumerate}[(a)]
    \item What is the probability of someone getting well, given that they receive the drug?    \vspace{1in}
    \item What is the probability that someone getting well, if they did not receive the drug? This is called \textit{the placebo effect.}  \vspace{1in}
\end{enumerate}
\end{example}
\smallskip 

Conditional probabilities are usually not reversible:
\[
P(A | B) \neq P(B | A)
\]
\smallskip 

\begin{example}
In a survey of pet owners, 45\% own a dog, 27\% own a cat, and 12\% own both a dog and a cat. 

\begin{enumerate}[(a)]
    \item What is the probability that a dog owner also owns a cat? \vspace{1in}
    \item What is the probability that a cat owner also owns a dog?
\end{enumerate}
\end{example}

\newpage 

Because $P(B|A) = \frac{P(A \text{ and } B)}{P(A)}$, then $P(A \text{ and } B) = P(A) \cdot P(B|A)$ \newline\\

You can use a tree diagram to help with conditional probability questions. \bigskip 

\begin{example}
A college reported the following based on their graduation data.
\begin{itemize}
    \item 70\% of freshmen had attended public schools
    \item 60\% of freshmen who had attended public schools graduated within 5 years
    \item 80\% of other freshmen graduated within 5 years
\end{itemize}
\bigskip 

What percent of freshmen graduated within 5 years?

\end{example}

\end{document}
