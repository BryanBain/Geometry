\documentclass{article}
\usepackage{amsmath, sfmath, multicol, tkz-euclide, array, enumerate, tcolorbox, tabularray, tipa}
\renewcommand{\familydefault}{\sfdefault}
\setlength{\parindent}{0cm}
\pagestyle{empty}
\usepackage[left=1in, top=0.5in, right=1in, bottom=0.5in]{geometry}
\tikzset{>=stealth, label style/.append style={font=\footnotesize}}
\tcbset{colback=white}

\newcounter{example}[section]
\newenvironment{example}[1][]{\refstepcounter{example}\par\medskip
   {\color{red}\textbf{Example~\theexample. #1}}}{\medskip}

\newcommand{\arc}[1]{%
    \setbox9=\hbox{#1}%
    \ooalign{\resizebox{\wd9}{\height}{\texttoptiebar{\phantom{A}}}\cr#1}}

\begin{document}

\section*{Permutations and Combinations}

\begin{tcolorbox}[colframe=orange!70!white, coltitle=black, title=\textbf{Today I Can}]
\begin{enumerate}
    \item Use permutations and combinations to solve problems.
\end{enumerate}
\end{tcolorbox}
\smallskip 

\begin{tcolorbox}[colframe=black!20!white, opacitybacktitle=0.1, coltitle=black, title=\textbf{Fundamental Counting Rule}]
If event $M$ can occur in $m$ ways and event $N$ occurs in $n$ ways, then event $M$ followed by event $N$ can occur in $m \cdot n$ ways.

\begin{itemize}
    \item Can also be used for situations that involve more than two events.
\end{itemize}
\end{tcolorbox}
\smallskip 

\begin{example}
A deli offers a lunch special if you choose one from each of the following types of sandwiches, side items, and drink choices. How many different lunch specials are possible?

\begin{center}
    \begin{tabular}{c|c|c}
        \textbf{Sandwiches} & \textbf{Side Items} & \textbf{Drinks} \\ \hline 
        ham \& turkey & chips & juice \\
        salami & potato salad & iced tea \\
        tuna & fruit salad & lemonade \\
        club & garden salad & milk \\ 
        veggie & & water \\ 
        meatball & & 
    \end{tabular}
\end{center}
\end{example}
\bigskip 

\begin{tcolorbox}[colframe=black!20!white, opacitybacktitle=0.1, coltitle=black, title=\textbf{Permutation}]
An arrangement of items in which \textbf{the order of the objects is important.} 

\begin{itemize}
    \item Can use factorial notation ($n!$) where
    \[
    n! = n(n-1)(n-2)\cdot \dots \cdot 3 \cdot 2 \cdot 1
    \]
    and $0! = 1$
\end{itemize}
\end{tcolorbox}
\smallskip 

\begin{example}
\begin{enumerate}[(a)]  \setlength{\itemsep}{30pt}
    \item You download 8 songs on your phone. If you play the songs using the random shuffle option, how many different ways can the sequence of songs be played?
    \item In how many ways can you arrange 12 books on a shelf?
\end{enumerate}
\end{example}
\vspace{30pt}

If you only want to play some (such as 3) of the 8 songs you downloaded, you have

\begin{center}
    8 ways to select the 1st song $\times$ 7 ways to select the 2nd song $\times$ 6 ways to select the 3rd song
\end{center}

\newpage

\begin{tcolorbox}[colframe=black!20!white, opacitybacktitle=0.1, coltitle=black, title=\textbf{Permutation Notation}]
The number of permutations of $n$ items of a set arranged $r$ items at a time is
\[
_nP_r = \frac{n!}{(n-r)!} \quad \text{for } 0 \leq r \leq n
\]
\smallskip 

\begin{itemize}
    \item \textbf{Example:} $_8P_3 = \frac{8!}{(8-3)!} = \frac{8!}{5!} = \frac{8 \cdot 7 \cdot 6 \cdot 5 \cdot 4 \cdot 3 \cdot 2 \cdot 1}{5 \cdot 4 \cdot 3 \cdot 2 \cdot 1} = 8 \cdot 7 \cdot 6 = 336$
\end{itemize}
\end{tcolorbox}
\smallskip 

\begin{example}
\begin{enumerate}[(a)]
    \item The environmental club is electing a president, vice president, a secretary, and treasurer. How many different ways can the officers be chosen from the 10 members?   \\[0.25in]
    \item Twelve swimmers compete in a race. In how many possible ways can the swimmers finish first, second, and third?
\end{enumerate}
\end{example}

\vspace{0.25in}

\begin{tcolorbox}[colframe=black!20!white, opacitybacktitle=0.1, coltitle=black, title=\textbf{Combination}]
A selection of items in which \textbf{order is not important.}
\[
_nC_r = \frac{n!}{r!(n-r)!} \quad \text{for } 0 \leq r \leq n
\]
\smallskip 

\begin{itemize}
    \item \textbf{Example:} $_9C_4 = \frac{9!}{4!(9-4)!} = \frac{9!}{4!5!} = \frac{9 \cdot 8 \cdot 7 \cdot 6 \cdot 5 \cdot 4 \cdot 3 \cdot 2 \cdot 1}{(4 \cdot 3 \cdot 2 \cdot 1)(5 \cdot 4 \cdot 3 \cdot 2 \cdot 1)} = \frac{9 \cdot 8 \cdot 7 \cdot 6}{4 \cdot 3 \cdot 2 \cdot 1} = 126$
\end{itemize}
\end{tcolorbox}
\smallskip 

\begin{example}
\begin{enumerate}[(a)]
    \item Suppose that you choose 4 books to read on summer vacation froma reading list of 12 books. How many different combinations of the books are possible? \vspace{0.25in}
    \item A sports club has 8 sophomores. Five of them are going to be chosen to go to a Browns game. How many different ways can the tickets be distributed?
\end{enumerate}
\end{example}
\vspace{0.35in}

\fbox{\fbox{To determine whether to use permutation or combination, you must decide whether order is important. }}
\smallskip 

\begin{example}
\begin{enumerate}[(a)]
    \item A \$100 prize, a \$50 prize, and a \$20 prize are to be given to 3 people who will be randomly chosen from a group of 15 people. How many ways can the prizes be given?   \\[0.35in]
    \item A jury of 12 people is to be chosen from a pool of 35 candidates. How many different juries are possible? 
\end{enumerate}
\end{example}

\vspace{0.35in}



\end{document}
