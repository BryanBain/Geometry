\documentclass{article}
\usepackage{amsmath, sfmath, multicol, tkz-euclide, array, enumerate, tcolorbox, tabularray, tipa}
\renewcommand{\familydefault}{\sfdefault}
\setlength{\parindent}{0cm}
\pagestyle{empty}
\usepackage[left=1in, top=0.5in, right=1in, bottom=0.5in]{geometry}
\tikzset{>=stealth, label style/.append style={font=\footnotesize}}
\tcbset{colback=white}

\newcounter{example}[section]
\newenvironment{example}[1][]{\refstepcounter{example}\par\medskip
   {\color{red}\textbf{Example~\theexample. #1}}}{\medskip}

\newcommand{\arc}[1]{%
    \setbox9=\hbox{#1}%
    \ooalign{\resizebox{\wd9}{\height}{\texttoptiebar{\phantom{A}}}\cr#1}}

\begin{document}

\section*{Compound Probability}

\begin{tcolorbox}[colframe=orange!70!white, coltitle=black, title=\textbf{Today I Can}]
\begin{enumerate}
    \item Identify independent and dependent events.
    \item Find compound probabilities.
\end{enumerate}
\end{tcolorbox}
\smallskip

\begin{tcolorbox}[colframe=black!20!white, opacitybacktitle=0.1, coltitle=black, title=\textbf{Compound Event}]
An event that is made up of 2 or more events.
\end{tcolorbox}
\smallskip

\begin{tcolorbox}[colframe=black!20!white, opacitybacktitle=0.1, coltitle=black, title=\textbf{Independent Events}]
When the occurrence of one event \textbf{does not affect} how another event occurs.
\end{tcolorbox}
\smallskip

\begin{tcolorbox}[colframe=black!20!white, opacitybacktitle=0.1, coltitle=black, title=\textbf{Dependent Events}]
When the occurrence of one event \textbf{does affect} how another event occurs.
\end{tcolorbox}
\smallskip

\begin{example}
Determine if the outcomes of each of the following trials are independent or dependent events.
\begin{enumerate}[(a)]
    \item Choose a number tile from 12 tiles. Then spin a spinner.  \\[0.25in]
    \item Pick one card from a standard deck of cards. Then, without replacing the card, pick another card. \\[0.25in]
    \item Pick one card from a standard deck of cards. Then, replace the card and pick another card.    \\[0.25in]
\end{enumerate}
\end{example}

\begin{tcolorbox}[colframe=black!20!white, opacitybacktitle=0.1, coltitle=black, title=\textbf{Probability of Independent Events}]
If $A$ and $B$ are independent events, then 
\[
P(A \text{ and } B) = P(A) \cdot P(B)
\]
\end{tcolorbox}
\smallskip

\begin{example}
A desk drawer contains 5 red pens, 6 blue pens, 3 black pens, 24 silver paper clips, and 16 gold paper clips. \newline 

If you select a pen and paper clip from the drawer without looking, find each probability.

\begin{enumerate}[(a)]
\begin{multicols}{2}
    \item Select a blue pen and gold paper clip.
    \item Select a red pen and a silver paper clip.
\end{multicols}
\end{enumerate}
\end{example}

\newpage 

\begin{tcolorbox}[colframe=black!20!white, opacitybacktitle=0.1, coltitle=black, title=\textbf{Mutually Exclusive Events}]
Events that cannot happen at the same time. \newline 

If $A$ and $B$ are mutually exclusive events, then 
\begin{itemize}
    \item $P(A \text{ and } B) = 0$
    \item $P(A \text{ or } B) = P(A) + P(B)$
\end{itemize}
\end{tcolorbox}
\smallskip

\begin{example}
Student athletes at a local high school may participate in only one sport each season. During the fall season, 28\% of student athletes play basketball and 24\% are on the swim team. What is the probability that a randomly selected student athlete plays basketball or is on the swim team?
\end{example}

\vspace{1in}

\begin{tcolorbox}[colframe=black!20!white, opacitybacktitle=0.1, coltitle=black, title=\textbf{Overlapping Events}]
Events that have outcomes in common. \newline 

If $A$ and $B$ are overlapping events, then 
\[
P(A \text{ or } B) = P(A) + P(B) - P(A \text{ and } B)
\]
\end{tcolorbox}
\smallskip

\begin{example}
A single die is rolled once. 

\begin{enumerate}[(a)]
    \item What is the probability of rolling either an even number or  a multiple of 3? \vspace{1.5in}
    \item What is the probability of rolling either an odd number or a number less than 4?
\end{enumerate}

\end{example}

\end{document}
