\documentclass{article}
\usepackage{amsmath, sfmath, multicol, tkz-euclide, array, enumerate, tcolorbox, tabularray, tipa}
\renewcommand{\familydefault}{\sfdefault}
\setlength{\parindent}{0cm}
\pagestyle{empty}
\usepackage[left=1in, top=0.5in, right=1in, bottom=0.5in]{geometry}
\tikzset{>=stealth, label style/.append style={font=\footnotesize}}
\tcbset{colback=white}

\newcounter{example}[section]
\newenvironment{example}[1][]{\refstepcounter{example}\par\medskip
   {\color{red}\textbf{Example~\theexample. #1}}}{\medskip}

\newcommand{\arc}[1]{%
    \setbox9=\hbox{#1}%
    \ooalign{\resizebox{\wd9}{\height}{\texttoptiebar{\phantom{A}}}\cr#1}}

\begin{document}

\section*{Probability Distributions and Frequency Tables}

\begin{tcolorbox}[colframe=orange!70!white, coltitle=black, title=\textbf{Today I Can}]
\begin{enumerate}
    \item Make and use frequency tables and probability distributions.
\end{enumerate}
\end{tcolorbox}
\smallskip 

\begin{tcolorbox}[colframe=black!20!white, opacitybacktitle=0.1, coltitle=black, title=\textbf{Frequency Table}]
A data display that shows how often an item appears in a category.
\end{tcolorbox}
\smallskip 

\begin{tcolorbox}[colframe=black!20!white, opacitybacktitle=0.1, coltitle=black, title=\textbf{Relative Frequency}]
The ratio of the frequency of the category to the total frequency. 
\begin{itemize}
    \item Relative frequency is often given as a percentage.
\end{itemize}
\end{tcolorbox}
\smallskip 

\begin{example}
The results of a survey of students' music preferences are organized in the frequency table below. Find the relative frequency for each of the following types of music.

\begin{center}
    \begin{tabular}{c|c|c|c|c|c|c}
        \textbf{Music Type} & Rock & Hip Hop& Country & Alternative & Classical & Other \\ \hline 
        \textbf{Frequency} & 10 & 7 & 8 & 5 & 6 & 4 
    \end{tabular}
\end{center}
\bigskip 

\begin{multicols}{4}
\begin{enumerate}[(a)]
    \item Rock
    \item Classical
    \item Hip Hop
    \item Country
\end{enumerate}
\end{multicols}
\end{example}

\vfill 

\begin{example}
A student conducts a probability experiment by tossing 3 coins at the same time. The results of each possible outcome, along with their frequencies, are shown below. Find the probability of each event.

\begin{center}
    \begin{tabular}{c|c|c|c|c|c|c|c|c}
        \textbf{Coin Toss Result} & HHH & HHT & HTT & HTH & THH & THT & TTH & TTT \\ \hline 
        \textbf{Frequency} & 5 & 7 & 9 & 6 & 2 & 9 & 10 & 2 
    \end{tabular}
\end{center}
\bigskip 

\begin{multicols}{2}
\begin{enumerate}[(a)]
    \item Exactly 2 tails occur
    \item No tails occur
\end{enumerate}
\end{multicols}
\end{example}

\vfill 
\newpage 

\begin{tcolorbox}[colframe=black!20!white, opacitybacktitle=0.1, coltitle=black, title=\textbf{Probability Distribution}]
A listing of each possible outcome along with its associated probability.
\begin{itemize}
    \item Often times shown in a frequency table.
\end{itemize}
\end{tcolorbox}
\bigskip 

\begin{example}
In a recent archery competition, 50 archers show 6 arrows each at a target. Three archers hit no bulls-eyes, 5 hit one bulls-eyes, 7 hit two bulls-eyes, 7 hit three bulls-eyes, 11 hit four bulls-eyes, 10 hit five bulls-eyes, and 7 hit six bulls-eyes. \newline

Fill in the table to create a probability distribution for the number of bulls-eyes each archer hit.
\bigskip 

\begin{center}
    \setlength{\extrarowheight}{10pt}
    \begin{tabular}{c|p{0.25in}|p{0.25in}|p{0.25in}|p{0.25in}|p{0.25in}|p{0.25in}|p{0.25in}}
        \textbf{Number of Bulls-Eyes} & & & & & & & \\[10pt] \hline 
        \textbf{Frequency} & & & & & & & \\[10pt] \hline
        \textbf{Probability} & & & & & & & \\[10pt] 
    \end{tabular}
\end{center}
\end{example}

\end{document}
